\documentclass{beamer}
\usepackage{graphicx}
\usepackage[T2A]{fontenc}
\usepackage[utf8]{inputenc}
\usepackage[russian]{babel}
\usepackage{ragged2e}
\usepackage{tikz}
\usepackage{pgfplots}
\pgfplotsset{compat=1.18}

\usetheme{Frankfurt}

\usecolortheme{}
%Information to be included in the title page:
\title{Орловский государственный университет имени И. С. Тургенева}
\author{Рудейко Е.Д.}
\institute{ОГУ им. И.С. Тургенева}
\date{2025}

\usebackgroundtemplate{
  \tikz[remember picture,overlay] \node[opacity=0.3, at=(current page.center)] {
    \includegraphics[width=\paperwidth,height=\paperheight]{download.jpg}
  };
}

\begin{document}

\frame{\titlepage}

\begin{frame}
\frametitle{Основание (1919-1931)}
{\centerline{\textbf{Образовании Орловского пролетарского университета.}}}
\centerline{Чуть позже 1935г. индустриально-педагогический институт.}
\includegraphics[width=1\textwidth]{haha.png}
\end{frame}

\begin{frame}
\frametitle{Ректора ОГУ им. Тургенева}
{\centerline{\textbf{Первый ректор Орловского пролетарского университета }}}
\centerline{Николай Иосифович Конрад}
\includegraphics[width=0.25\textwidth]{5.jpg}
\includegraphics[width=0.6\textwidth]{6.jpg}
\end{frame}

\begin{frame}
\frametitle{Первые ученики (1931-1941)}
\centerline{\textit{В первые года было: 121 зачисленных, 4 факультета, 11 преподавателей.}}
\centerline{Мужская гимназия становление её в Университет.}
\includegraphics[width=1.04\textwidth\center]{7.jpg}
\end{frame}



\begin{frame}
\frametitle{Война (1941-1945)}
\centerline{Многие ушли на фронт.}
\centerline{Эвакуация в Бирск (1941) и Елец (1943).}
\includegraphics[width=1.04\textwidth]{9 (2).jpg}
\end{frame}

\begin{frame}
\frametitle{Послевоенное развитие}
\includegraphics[width=0.9\textwidth \center]{pfpfp.JPG}
\centerline{Переезд на Комсомольскую, 95 (1957).}
\centerline{Подготовка иностранных студентов (с 1980-х).}
\end{frame}

\begin{frame}
\frametitle{Статус университета и присвоение имени И.С. Тургенева}
\centerline{Статус классического университета (1996).}
\centerline{Присвоение имени И.С. Тургенева (2014).}
\includegraphics[width=1\textwidth]{12.jpg}
\end{frame}
\begin{frame}
  \frametitle{Этапы истории ОГУ}
  \begin{table}[h!]
\centering
 \begin{tabular}{||c c||} 
 \hline
 Дата & Описание \\ [0.5ex] 
 \hline\hline
 1919 год & Основание Орловского Института \\ 
 1931 год & Образование Орловского Пед. Института\\
 19(41-44) & Период Войны  Эвакуация в Башкирскую ССР \\
 1952 год & Преобразование в ОПТУ  \\
 1981 год & Награждение "Знаком почета"  \\
 1996 гож & Получил статус классического ВУЗа \\ [1ex] 
 \hline
 \end{tabular}
\end{table}

\end{frame}

\end{document}
