\documentclass{article}
\usepackage{cmap} % Включает CMap-таблицу в русскоязычный PDF, чтобы функции поиска и копирования работали правильно
\usepackage[T2A]{fontenc} % загружает кодировки шрифтов, T1 для латиницы и T2A для кириллицы
%\usepackage[utf8]{inputenc} % загружает кодировку UTF-8
\usepackage[english, russian]{babel} % загружает поддержку языков
\usepackage{textcase}
\usepackage{bbding}
\usepackage{titlesec}
\usepackage{multirow}
\usepackage{derivative}
\usepackage{bigints}
\usepackage{graphics,bm,array}
\usepackage{graphicx}
\usepackage{caption}
\usepackage{amsmath}
\begin{document}
\title{}
\date{}
\maketitle{}
\par
позволяет привести уравнение (21) к более простому виду

\[
U_{x x}U_{y y}+c_1U=0, c_1=\frac{1}{2}(4c^2-a^2-b^2),-\infty<x<\infty,y>0)\] \begin{flushright}
    (25)
\end{flushright}

с дополнительными условиями

\[
U|_{y=0}=\phi(x)e^{\frac{a}{2} \cdot x}=\phi_1(x), -\infty<x<\infty\] \begin{flushright}
    (22$'$)
\end{flushright}

\[
U_y|_{y=0}=(\psi(x)-\phi\frac{b}{2}(x))e^{\frac{a}{2} \cdot x}=\psi_1(x), -\infty<x<\infty\] \begin{flushright}
    (23$'$)
\end{flushright}
если только выбрать параметры $\lambda$ и $\mu$ соответствующим образом, 

полагая
\[\ \lambda=\frac{a}{2}, \mu=-\frac{b}{2}  \]
\begin{flushright}
    (26)
\end{flushright}
Определение функции $U($x$,$у$)$ по начальным данным и уравнению (25) сводится к построению функции Римана $v(x, y; \xi, \eta)$.

Функция $v$ должна удовлетворять условиям:

\[\ v_{xx}-v_{xx}+c_1v=0,\]
\begin{flushright}
    (27)
\end{flushright}
\begin{center}
$
    \begin{cases}
        v=1\quad $на\quad характеристике\quad MP,$ \\
        v=1\quad $на\quad характеристике\quad MQ(рис. 28)$.
    \end{cases}
$
\end{center}
\begin{flushright}
    (28)
\end{flushright}

Будем искать $v$ в виде

\[v=v(z)\]
\begin{flushright}
    (29)
\end{flushright}
где 

$z=\sqrt{(x-\xi)^2-(y-\eta)^2}$ или $z^2=(x-\xi)^2-(y-\eta)^2,$
\begin{flushright}
    (30)
\end{flushright}

На характеристиках $MP$ и $MQ$ переменная $z$ обращается в нуль, так что $v(0)=1$. Далее, левая часть уравнения (27) преобразуется следующим образом:

\[
v_{x x}-v_{y y}+c_1v=v''(z)(z^2_x-z^2_y)+v'(z)(z_{x x}-z_{y y})+c_1v=0\]

Дифференцируя выражение для $z^2$ дважды, по $x$ и $y$, получим:
\[z_{xx}=x-\xi,\]
\[zz_y=-(y-\eta),\]
\[zz_{x x}+z^2_x=1,\]
\[zz_{yy}+z^2_y=-1.\]

Отсюда и из формулы (30) находим:

\[z^2_x-z^2_y=1,z_{xx}-z_{yy}=\frac{1}{z}\]

Уравнение для $v$ принимает следующий вид:

\[v''+\frac{1}{z}v'+c_1v=0\]


\maketitle{}
\par
при условии и $v$(0) = 1. Решением этого уравнения является функция Бесселя нулевого порядка (см. Дополнение ІІ, часть I,  §1)

\[v(z)=J_0(\sqrt{c_1}z)\]

или

\[v(x,y;\xi,\eta)=J_0(\sqrt{c_1[(x-\xi)^2-(y-\eta)]}).\]
\begin{flushright}
    (31)
\end{flushright}
Воспользуемся теперь для нахождения $U$$(x,y)$ формулой (10), которая в нашем случае принимает вид 


\[U(M) = \frac{U(P) + U(Q)}{2} + \frac{1}{2} \int_P^Q \big(vU_\eta d\xi - U_\eta vd\xi \big) \quad (d\eta = 0).\]
\begin{flushright}
    (32)
\end{flushright}

Вычислим предварительно интеграл по отрезку $PQ$ $(\eta=0)$


\[\int_P^Q \big(vU_\eta - U_\eta v \big)d\xi = \int_{x-y}^{x+y} \bigg\{ J_0 \big( \sqrt{V_{c1} \big((x - \xi)^2 - y^2\big)} \big) U_\eta (\xi, 0) -\]
\[-\frac{U_{(\xi, 0)} \sqrt{c_1} \, J_0'(\sqrt{c_1}\sqrt{(x=e)^2-y^2})}{\sqrt{c_1\big[(x - \xi)^2 - y^2}} \bigg\} d\xi\] 
\begin{flushright}
    (33)
\end{flushright}

Пользуясь начальными условиями (22$'$), (23$'$), находим:


\[U(x, y) = \frac{\varphi_1(x - y) + \varphi_1(x + y)}{2} + \frac{1}{2}\int_{x-y}^{x+y} J_0 \big( \sqrt{c1} \sqrt{(x - \xi)^2 - y^2 \big)}  \Psi_1(\xi) d\xi+ \]
\[\quad + \frac{1}{2} \sqrt{c_1}\, y \int_{x-y}^{x+y} \frac{J_1 ( \sqrt{c_1} \sqrt{(x - \xi)^2 - y^2} \big) \varphi_1(\xi)d\xi}{\sqrt{(x - \xi)^2 - y^2}}.\]
\begin{flushright}
    (34)
\end{flushright}

откуда в силу (24), (22$'$) и (23$'$) получаем формулу

\[u(x, y) = \frac{\varphi(x - y) e^{-\frac{a - b}{2} y} + \varphi(x + y) e^{\frac{a + b}{2} y}}{2}- \]
\[ - \frac{1}{2} e^{\frac{b}{2} y}  \int_{x-y}^{x+y} \bigg\{ \frac{b}{2} J_0 (\sqrt{c_1} \sqrt{(x - \xi)^2 - y^2} - \]
\[ - \sqrt{c_1}\, y \frac{J_1 \big( \sqrt{c_1} \sqrt{(x - \xi)^2 - y^2}}{\sqrt{\big(x - \xi)^2 - y^2}} \bigg\} e^{-\frac{a}{2} (x - \xi)}\varphi(\xi) d\xi \]
\[\quad + \frac{1}{2} e^{\frac{b}{2} y} 
\int_{x-y}^{x+y} J_0 ( \sqrt{c_1} \sqrt{(x - \xi)^2 - y^2} \big) e^{-\frac{a}{2} (x - \xi)} \Psi(\xi) d\xi.\]
\begin{flushright}
    (35)
\end{flushright}


дающее решение поставленной задачи.
\end{document}